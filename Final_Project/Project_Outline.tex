\documentclass[12pt]{article}
%\documentstyle[12pt]{article}
\setlength{\oddsidemargin}{0in}
\setlength{\evensidemargin}{0in}
\setlength{\textwidth}{6.5in}
\setlength{\topmargin}{-.3in}
\setlength{\textheight}{9in}
\pagestyle{empty}

\usepackage[super]{nth}
\usepackage{amsmath}
\usepackage{csquotes}
\usepackage{physics}
\usepackage{graphicx}% Include figure files
\usepackage{dcolumn}% Align table columns on decimal point
\usepackage{bm}

\begin{document}

\begin{center}
{\Large Final Project Research Proposal} \\[.3in]
{\large Bj\"{o}rn Sumner and Ben Crane} \\
{21 March 2018}
\end{center}

\section*{Absorption imaging of atoms in a magneto-optical trap}

\subsection*{Overview}
For our final project, we propose to design and implement an absorption imaging setup to measure the temperature of atoms in a magneto-optical trap (MOT).  We will use the current setup for the atom trapping experiment and combine it with the laser spectroscopy equipment to create a probe beam for atoms in the MOT.  This probe beam will be collected and analyzed by either a photodetector or a camera, depending on available equipment.

\subsection*{Detail}

There are several components that will require construction in order to implement our experiment:

We will need to construct a means of shutting off the magnetic field as rapidly as possible.

We will need to 





 and its intensity measured.  Our goal is to provide a measure of the temperature of atoms within the MOT by analyzing the expansion of the atoms in time once confinement is released.  This will allow us to optimize the MOT and attempt to realize the Doppler cooling limit by observing the temperature as various parameters of the MOT, such as trapping laser intensity, are adjusted.


\end{document}
